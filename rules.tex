\documentclass{article}

\usepackage[margin=0.5in]{geometry}
\setlength\parindent{0pt}
\title{3D Printer Rules and Guidelines}
\author{Neve Laughery}

\begin{document}

\maketitle

\section{Administration}

The 3D printing administrator will be chosen each semester by the Edinboro 
Computer Science executive board. They will be responsible for the following:

\begin{itemize}
	\item Maintaining the 3D printer
	\begin{itemize}
		\item Calibrating the machine when needed
		\item Ordering filament
		\item Researching and solving technical issues related to the machine
	\end{itemize}
	\item Discussing viability of projects with users
	\item Managing each user's print quota
	\item Creating time slots to avoid scheduling conflicts
\end{itemize}

\section{Users}

Failure to follow these rules and guidelines will result in TBA

\subsection{Eligibility}

Users should be members of the Edinboro University Computer Science Club.
Members of this club will be given priority scheduling over non-members. Prints
requested by non-member Edinboro University students will be approved or
declined at the discretion of the administrator. In both cases the user must 
discuss their project with the administrator before printing.

\subsection{Print Quota}

Users will be allotted TBA prints per academic year, this will be tracked by the
administrator. If a user wishes to exceed this quota they must first get
express permission by the administrator.

\subsection{Filament Used}

Users will weigh their projects, including failed prints, on the scale located
next to the printer. Users will pay TBA per gram of filament used to the
Edinboro Computer Science Club Treasurer. This amount is slightly more than the
cost of the filament to account for wear and tear on the machine.

If a user wants to use their own filament for a print it should be discussed
with the administrator first.

The Creality Ender 3 can print the following filaments in 1.75 mm:

\begin{itemize}
	\item PLA
	\item TPU
	\item ABS
\end{itemize}

\subsection{Failed Prints}

Users are responsible for their own failed prints. Should a print fail the user
can either consult the Troubleshooting document or request help from the
administrator. Users must pay for all filament used including failed prints.

\end{document}
